\newpage
\section{Appendix}
\label{sec:app}
\subsection{\texttt{IBST.h}}
\begin{lstlisting}
#pragma once
#include <cstddef>

template<class Key>
class IBST {
public:
    virtual void insert(const Key& k)                = 0;
    virtual bool contains(const Key& k)        const = 0;
    virtual std::size_t size_bytes()           const = 0;
    virtual ~IBST() = default;
};
\end{lstlisting}
\captionof{lstlisting}[\texttt{IBST.h}]{Abstract interface class for our Binary Search Tree implementationin in \texttt{IBST.h}}
\label{lst:ibst}

\subsection{\texttt{BSTPtr.h}}
\begin{lstlisting}
#pragma once
#include "IBST.h"
#include <memory>
#include <utility>

template<class Key>
class BSTPtr : public IBST<Key> {
    struct Node {
        Key key;
        std::unique_ptr<Node> l, r;
        explicit Node(const Key& k) : key(k) {}
    };

    std::unique_ptr<Node> root_;
    std::size_t node_cnt_ = 0;

    void insertNode(std::unique_ptr<Node>& p, const Key& k) {
        if (!p) {
            p = std::make_unique<Node>(k);
            ++node_cnt_;                         
        } else if (k < p->key) {
            insertNode(p->l, k);
        } else if (k > p->key) {
            insertNode(p->r, k);
        }
    }

    static bool contains(const std::unique_ptr<Node>& p, const Key& k) {
        const Node* cur = p.get();
        while (cur) {
            if (k == cur->key) return true;
            cur = (k < cur->key) ? cur->l.get() : cur->r.get();
        }
        return false;
    }

public:
    void insert(const Key& k) override { insertNode(root_, k); }
    bool contains(const Key& k) const override { return contains(root_, k); }

    std::size_t size_bytes() const override { return node_cnt_ * sizeof(Node); }
};
\end{lstlisting}
\captionof{lstlisting}[\texttt{BSTPtr.h}]{Plain Binary Search Tree implementation in \texttt{BSTPtr.h}}
\label{lst:btspointer}

\subsection{\texttt{BSTVEB.h}}
\label{secsec:vb}
\begin{lstlisting}
#pragma once
#include "IBST.h"
#include <vector>
#include <algorithm>
#include <stdexcept>


namespace help {
template<class Key>
void build_veb(std::vector<Key>& out,
               const std::vector<Key>& sorted, std::size_t lo, std::size_t hi)
{
    if (lo >= hi) return;                       
    std::size_t mid = (lo + hi) / 2;            
    out.push_back(sorted[mid]);                 
    build_veb(out, sorted, lo, mid);            
    build_veb(out, sorted, mid + 1, hi);        
}
} 

template<class Key>
class BSTVEB : public IBST<Key> {
    std::vector<Key> a_;            
    bool              frozen_ = false;
    std::vector<Key>  inserts_;     


    void freeze() {
        if (frozen_) return;

        std::sort(inserts_.begin(), inserts_.end());
        inserts_.erase(std::unique(inserts_.begin(), inserts_.end()),
                       inserts_.end());

        a_.reserve(inserts_.size());
        help::build_veb(a_, inserts_, 0, inserts_.size());
        frozen_ = true;
    }

    bool containsRec(const Key& k,
                     std::size_t lo, std::size_t hi, std::size_t idx) const
    {
        if (lo >= hi) return false;            

        const Key& key = a_[idx];
        if (k == key) return true;

        std::size_t mid        = (lo + hi) / 2;
        std::size_t left_size  = mid - lo;    
        std::size_t left_idx   = idx + 1;   
        std::size_t right_idx  = idx + 1 + left_size;

        return (k < key)
             ? containsRec(k, lo, mid,           left_idx)
             : containsRec(k, mid + 1, hi,       right_idx);
    }

public:
    void insert(const Key& k) override {
        if (frozen_)
            throw std::logic_error("I am already frozen!");
        inserts_.push_back(k);
    }

    bool contains(const Key& k) const override {
        const_cast<BSTVEB*>(this)->freeze();   
        return containsRec(k, 0, a_.size(), 0);
    }

     std::size_t size_bytes() const override { return a_.size() * sizeof(Key); }
};

\end{lstlisting}
\captionof{lstlisting}[\texttt{BSTVEB.h}]{Van Emde-Boas Tree Implementation in \texttt{BSTVEB.h}}
\label{lst:btsvb}

\subsection{\texttt{benchmark.cpp}}
\label{secsec:bcpp}
\begin{lstlisting}
#include "IBST.h"
#include "BSTPtr.h"
#include "BSTVEB.h"
#include "PerfCounters.h"

#include <vector>
#include <random>
#include <iostream>
#include <iomanip>
#include <fstream>
#include "util/json.hpp"
#include <functional>
#include <chrono>

using json  = nlohmann::json;
using Clock = std::chrono::steady_clock;

struct Metrics {
    long long ns          = 0;
    long long ops         = 0;
    long long cache_refs  = 0;
    long long cache_miss  = 0;
};

template<class Key>
Metrics benchOnce(IBST<Key>& tree,
                  const std::vector<Key>& lookups,
                  const std::vector<Key>& inserts)
{
    for (const auto& k : inserts) tree.insert(k);

    PerfCounters pc;
    pc.start();

    auto start = Clock::now();
    for (const auto& k : lookups)
        (void)tree.contains(k);
    auto end   = Clock::now();

    pc.stop();

    Metrics m;
    m.ns         = std::chrono::duration_cast<std::chrono::nanoseconds>(end - start).count();
    m.ops        = lookups.size();
    m.cache_refs = pc.refs();
    m.cache_miss = pc.misses();
    return m;
}

using Factory = std::function<std::unique_ptr<IBST<int>>(void)>;
struct Variant { std::string name; Factory make; };

const std::vector<Variant> variants = {
    {"BST_PTR", [] { return std::make_unique<BSTPtr<int>>(); }},
    {"BST_VEB", [] { return std::make_unique<BSTVEB<int>>(); }},
};

void runExperiment(int n, int q, int T, bool csv,
                   const Factory& make, unsigned seed,
                   const std::string& impl)
{
    std::mt19937 rng(seed);
    std::uniform_int_distribution<int> dist(1, n * 10);

    std::vector<int> inserts(n);
    for (int& x : inserts) x = dist(rng);

    std::vector<int> lookups(q);
    for (int& x : lookups) x = dist(rng);

    long long  acc_ns   = 0;
    long long  acc_refs = 0, acc_miss = 0;
    std::size_t bytes_used = 0;

    for (int t = 0; t < T; ++t) {
        auto tree = make();
        Metrics m = benchOnce(*tree, lookups, inserts);
        acc_ns   += m.ns;
        acc_refs += m.cache_refs;
        acc_miss += m.cache_miss;
        if (t == 0) bytes_used = tree->size_bytes();  
    }

    double avg_ns      = double(acc_ns)   / T;
    double ns_per_op   = avg_ns / q;

    double avg_refs    = double(acc_refs) / T;
    double avg_miss    = double(acc_miss) / T;
    double miss_per_op = avg_miss / q;
    double miss_rate   = (avg_refs > 0) ? avg_miss / avg_refs : 0.0;  
    double avg_s       = avg_ns / 1e9;
    double bytes_mb    = bytes_used / 1024.0 / 1024.0;

    if (csv) {
        std::cout << impl << ','
                  << n << ',' << q << ','
                  << avg_ns << ','
                  << std::fixed << std::setprecision(2)
                  << avg_s << ','
                  << std::defaultfloat
                  << ns_per_op << ','
                  << avg_refs << ',' << avg_miss << ','
                  << miss_per_op << ','
                  << miss_rate << ','
                  << bytes_used  << '\n';
    } else {
        std::cout << std::fixed << std::setprecision(2)
                  << std::left
                  << std::setw(10) << impl
                  << std::setw(10) << n
                  << std::setw(10) << q
                  << std::setw(15) << avg_ns
                  << std::setw(10) << avg_s
                  << std::defaultfloat << std::setprecision(6)
                  << std::setw(15) << ns_per_op
                  << std::setw(15) << avg_refs
                  << std::setw(15) << avg_miss
                  << std::setw(12) << miss_per_op
                  << std::setw(10) << std::fixed << std::setprecision(3) << miss_rate
                  << std::setw(12) << std::fixed << std::setprecision(1) << bytes_mb
                  << '\n';
    }
}

int main(int argc, char* argv[])
{
    int         n    = 10000;
    int         q    = 10000;
    int         T    = 1;
    bool        csv  = false;
    unsigned    seed = 42;
    std::string impl = "ALL";

    if (argc >= 2) {
        std::ifstream in(argv[1]);
        if (!in) { std::cerr << "Cannot open " << argv[1] << '\n'; return 1; }
        json cfg; in >> cfg;
        if (cfg.contains("n"))    n    = cfg["n"];
        if (cfg.contains("q"))    q    = cfg["q"];
        if (cfg.contains("T"))    T    = cfg["T"];
        if (cfg.contains("csv"))  csv  = cfg["csv"];
        if (cfg.contains("seed")) seed = cfg["seed"];
        if (cfg.contains("impl")) impl = cfg["impl"];
    }

    if (argc == 3) impl = argv[2];

    if (!csv) {
        std::cout << std::left
                  << std::setw(10) << "impl"
                  << std::setw(10) << "n"
                  << std::setw(10) << "q"
                  << std::setw(15) << "total_ns"
                  << std::setw(10) << "total_s"
                  << std::setw(15) << "ns/search"
                  << std::setw(15) << "cache_refs"
                  << std::setw(15) << "cache_miss"
                  << std::setw(12) << "miss/search"
                  << std::setw(10) << "missRate"
                  << std::setw(12) << "bytes(MB)" << '\n'
                  << std::string(134, '-') << '\n';
    } else {
        std::cout << "impl,n,q,total_ns,total_s,ns_per_search,"
                     "cache_refs,cache_misses,misses_per_search,miss_rate,bytes\n";
    }

    for (const auto& v : variants) {
        if (impl != "ALL" && impl != v.name) continue;
        runExperiment(n, q, T, csv, v.make, seed, v.name);
    }
    return 0;
}

\end{lstlisting}
\captionof{lstlisting}[\texttt{BSTVEB.h}]{Benchmarking Driver in \texttt{benchmark.cpp}}
\label{lst:benchcpp}

\subsection{\texttt{PerfCounters.h}}
\label{secsec:perfsetup}
\begin{lstlisting}
#pragma once
#include <linux/perf_event.h>
#include <cstdint>         
#include <sys/syscall.h>    
#ifndef __NR_perf_event_open
#endif
#include <sys/ioctl.h>
#include <unistd.h>
#include <cstring>
#include <cerrno>
#include <stdexcept>
#include <array>

class PerfCounters {
    int fd_refs_{-1}, fd_misses_{-1};
    long long refs_{0}, misses_{0};

    static int openCounter(uint32_t type, uint64_t config)
    {
        perf_event_attr pea{};
        pea.type           = type;
        pea.size           = sizeof(perf_event_attr);
        pea.config         = config;
        pea.inherit        = 0;       
        pea.disabled       = 1;
        pea.exclude_kernel = 0;
        pea.exclude_hv     = 1;

        int fd = syscall(__NR_perf_event_open, &pea, 0, -1, -1, 0);
        if (fd == -1)
            throw std::runtime_error{"perf_event_open: " + std::string(strerror(errno))};
        return fd;
    }
public:
    PerfCounters()
    {
        fd_refs_   = openCounter(PERF_TYPE_HARDWARE, PERF_COUNT_HW_CACHE_REFERENCES);
        fd_misses_ = openCounter(PERF_TYPE_HARDWARE, PERF_COUNT_HW_CACHE_MISSES);
    }
    ~PerfCounters() { close(fd_refs_); close(fd_misses_); }

    inline void start() const { ioctl(fd_refs_, PERF_EVENT_IOC_RESET, 0); ioctl(fd_misses_, PERF_EVENT_IOC_RESET, 0);
                                ioctl(fd_refs_, PERF_EVENT_IOC_ENABLE, 0); ioctl(fd_misses_, PERF_EVENT_IOC_ENABLE, 0); }
    inline void stop()
{
    ioctl(fd_refs_,   PERF_EVENT_IOC_DISABLE, 0);
    ioctl(fd_misses_, PERF_EVENT_IOC_DISABLE, 0);

    if (read(fd_refs_, &refs_, sizeof(refs_))   != sizeof(refs_))
        throw std::runtime_error("failed to read cache-refs counter");
    if (read(fd_misses_, &misses_, sizeof(misses_)) != sizeof(misses_))
        throw std::runtime_error("failed to read cache-miss counter");
}
    long long refs()   const { return refs_;   }
    long long misses() const { return misses_; }
};

\end{lstlisting}
\captionof{lstlisting}[\texttt{PerfCounters.h}]{Instrumentation Setup for Perf in \texttt{PerfCounters.h}}
\label{lst:perfsetup}


\subsection{\texttt{small.json}}
\label{secsec:smalinstance}
\begin{lstlisting}
{
  "n"   : 100000,
  "q"   : 100000,
  "T"   : 5,
  "csv" : false,
  "seed": 123
}
\end{lstlisting}
\captionof{lstlisting}[\texttt{small.json}]{Example Instance in \texttt{small.json}}
\label{lst:exampleinstance}