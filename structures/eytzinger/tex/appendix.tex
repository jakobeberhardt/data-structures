\newpage
\section{Appendix}
\label{sec:app}

\subsection{\texttt{include/IBST.h}}
\label{secsec:ibst-h}
\begin{lstlisting}
#pragma once
#include <cstddef>

template<class Key>
class IBST {
public:
    virtual void insert(const Key& k)                = 0;
    virtual bool contains(const Key& k)        const = 0;
    virtual std::size_t size_bytes()           const = 0;
    virtual ~IBST() = default;
};
\end{lstlisting}
\captionof{lstlisting}[\texttt{IBST.h}]{Tiny pure-virtual interface all BST
variants must implement.}
\label{lst:ibst-h}

\subsection{\texttt{include/BSTVEB.h}}
\label{secsec:bstveb-h}
\begin{lstlisting}
#pragma once
#include "IBST.h"
#include <vector>
#include <algorithm>
#include <stdexcept>

namespace help {
template<class Key>
void build_veb(std::vector<Key>& out,
               const std::vector<Key>& sorted, std::size_t lo, std::size_t hi)
{
    if (lo >= hi) return;                       
    std::size_t mid = (lo + hi) / 2;            
    out.push_back(sorted[mid]);                 
    build_veb(out, sorted, lo, mid);            
    build_veb(out, sorted, mid + 1, hi);        
}
} 

template<class Key>
class BSTVEB : public IBST<Key> {
    std::vector<Key> a_;            
    bool              frozen_ = false;
    std::vector<Key>  inserts_;     


    void freeze() {
        if (frozen_) return;

        std::sort(inserts_.begin(), inserts_.end());
        inserts_.erase(std::unique(inserts_.begin(), inserts_.end()),
                       inserts_.end());

        a_.reserve(inserts_.size());
        help::build_veb(a_, inserts_, 0, inserts_.size());
        frozen_ = true;
    }

    bool containsRec(const Key& k,
                     std::size_t lo, std::size_t hi, std::size_t idx) const
    {
        if (lo >= hi) return false;            

        const Key& key = a_[idx];
        if (k == key) return true;

        std::size_t mid        = (lo + hi) / 2;
        std::size_t left_size  = mid - lo;    
        std::size_t left_idx   = idx + 1;   
        std::size_t right_idx  = idx + 1 + left_size;

        return (k < key)
             ? containsRec(k, lo, mid,           left_idx)
             : containsRec(k, mid + 1, hi,       right_idx);
    }

public:
    void insert(const Key& k) override {
        if (frozen_)
            throw std::logic_error("I am already frozen!");
        inserts_.push_back(k);
    }

    bool contains(const Key& k) const override {
        const_cast<BSTVEB*>(this)->freeze();   
        return containsRec(k, 0, a_.size(), 0);
    }

     std::size_t size_bytes() const override { return a_.size() * sizeof(Key); }
};
\end{lstlisting}
\captionof{lstlisting}[\texttt{BSTVEB.h}]{Van Emde-Boas layout search-tree
implementation.}
\label{lst:bstveb-h}


\subsection{\texttt{include/BSTEyt.h}}
\label{secsec:bsteyt-h}
\begin{lstlisting}
#pragma once
#include "IBST.h"
#include <vector>
#include <algorithm>
#include <stdexcept>

template<class Key>
class BSTEyt : public IBST<Key> {
protected:  
    std::vector<Key> arr_; 
    static void dedupSort(std::vector<Key>& v) {
        std::sort(v.begin(), v.end());
        v.erase(std::unique(v.begin(), v.end()), v.end());
    }

    void buildEyt(std::size_t idx, std::size_t& pos,
                  const std::vector<Key>& sorted)
    {
        if (idx >= sorted.size()) return;
        buildEyt(2*idx + 1, pos, sorted);
        arr_[idx] = sorted[pos++];
        buildEyt(2*idx + 2, pos, sorted);
    }

    void freeze()
    {
        if (frozen_) return;
        dedupSort(inserts_);
        arr_.resize(inserts_.size());
        std::size_t p = 0;
        buildEyt(0, p, inserts_);
        frozen_ = true;
        inserts_.clear();
        inserts_.shrink_to_fit();
    }

private:
    std::vector<Key> inserts_;           
    bool             frozen_ = false;

public:
    void insert(const Key& k) override {
        if (frozen_)
            throw std::logic_error("BST_EYT: insert after first query");
        inserts_.push_back(k);
    }

    bool contains(const Key& k) const override {
        const_cast<BSTEyt*>(this)->freeze();

        std::size_t i = 0;
        while (i < arr_.size()) {
            if (k == arr_[i])           return true;
            i = (k < arr_[i]) ? 2*i + 1 : 2*i + 2;
        }
        return false;
    }

    std::size_t size_bytes() const override {
        return arr_.size() * sizeof(Key);
    }
};
\end{lstlisting}
\captionof{lstlisting}[\texttt{BSTEyt.h}]{Eytzinger-layout binary-search-tree
base class (\texttt{BSTEyt}).}
\label{lst:bsteyt-h}


\subsection{\texttt{include/BSTEytPrefetch.h}}
\label{secsec:bsteyt-pref-h}
\begin{lstlisting}
#pragma once
#include "BSTEyt.h"
#include <cstddef>

template<class Key>
class BSTEytPref : public BSTEyt<Key> {
    using Base = BSTEyt<Key>;

public:
    bool contains(const Key& k) const override {
        const_cast<BSTEytPref*>(this)->Base::freeze();

        const auto& a = Base::arr_;   
        std::size_t i = 0;

        while (i < a.size()) {
            std::size_t l = 2*i + 1;
            std::size_t r = l + 1;
            if (l < a.size()) __builtin_prefetch(&a[l], 0, 1);
            if (r < a.size()) __builtin_prefetch(&a[r], 0, 1);

            if      (k == a[i]) return true;
            else if (k <  a[i]) i = l;
            else                i = r;
        }
        return false;
    }
};
\end{lstlisting}
\captionof{lstlisting}[\texttt{BSTEytPrefetch.h}]{Single-level software
prefetch variant.}
\label{lst:bsteyt-pref-h}



\subsection{\texttt{include/BSTEytPrefetchTwo.h}}
\label{secsec:bsteyt-pref-two-h}
\begin{lstlisting}
#pragma once
#include "BSTEyt.h"
#include <cstddef>

template<class Key>
class BSTEytPrefTwo : public BSTEyt<Key> {
    using Base = BSTEyt<Key>;

public:
    bool contains(const Key& k) const override {
        const_cast<BSTEytPrefTwo*>(this)->Base::freeze();
        const auto& a = Base::arr_;

        std::size_t i = 0;
        while (i < a.size()) {
            std::size_t l  = 2*i + 1;
            std::size_t r  = l + 1;
            if (l < a.size()) __builtin_prefetch(&a[l], 0, 1);
            if (r < a.size()) __builtin_prefetch(&a[r], 0, 1);

            std::size_t ll = 4*i + 3;   
            std::size_t lr = ll + 2;    
            std::size_t rl = 4*i + 5;   
            std::size_t rr = rl + 2;    

            if (ll < a.size()) __builtin_prefetch(&a[ll], 0, 1);
            if (lr < a.size()) __builtin_prefetch(&a[lr], 0, 1);
            if (rl < a.size()) __builtin_prefetch(&a[rl], 0, 1);
            if (rr < a.size()) __builtin_prefetch(&a[rr], 0, 1);

            if      (k == a[i]) return true;
            else if (k <  a[i]) i = l;
            else                i = r;
        }
        return false;
    }
};
\end{lstlisting}
\captionof{lstlisting}[\texttt{BSTEytPrefetchTwo.h}]{Two-level software
prefetch variant.}
\label{lst:bsteyt-pref-two-h}


\subsection{\texttt{include/BSTEytPrefetchThree.h}}
\label{secsec:bsteyt-pref-three-h}
\begin{lstlisting}
#pragma once
#include "BSTEyt.h"
#include <cstddef>

template<class Key>
class BSTEytPrefThree : public BSTEyt<Key> {
    using Base = BSTEyt<Key>;

public:
    bool contains(const Key& k) const override
    {
        const_cast<BSTEytPrefThree*>(this)->Base::freeze();
        const auto& a = Base::arr_;

        auto pf = [&](std::size_t idx) {
            if (idx < a.size()) __builtin_prefetch(&a[idx], 0, 1);
        };

        std::size_t i = 0;
        while (i < a.size()) {

            std::size_t l = 2*i + 1;
            std::size_t r = l + 1;
            pf(l); pf(r);

            std::size_t ll = 2*l + 1;
            std::size_t lr = ll + 1;
            std::size_t rl = 2*r + 1;
            std::size_t rr = rl + 1;
            pf(ll); pf(lr); pf(rl); pf(rr);

            pf(2*ll + 1); pf(2*ll + 2);
            pf(2*lr + 1); pf(2*lr + 2);
            pf(2*rl + 1); pf(2*rl + 2);
            pf(2*rr + 1); pf(2*rr + 2);

            if      (k == a[i]) return true;
            else if (k <  a[i]) i = l;
            else                i = r;
        }
        return false;
    }
};
\end{lstlisting}
\captionof{lstlisting}[\texttt{BSTEytPrefetchThree.h}]{Three-level software
prefetch variant.}
\label{lst:bsteyt-pref-three-h}


\subsection{\texttt{include/BSTEytPrefetchFour.h}}
\label{secsec:bsteyt-pref-four-h}
\begin{lstlisting}
#pragma once
#include "BSTEyt.h"
#include <cstddef>

template<class Key>
class BSTEytPrefFour : public BSTEyt<Key> {
    using Base = BSTEyt<Key>;

    static inline void pf(const std::vector<Key>& a, std::size_t idx)
    {
        if (idx < a.size()) __builtin_prefetch(&a[idx], 0, 1);
    }

public:
    bool contains(const Key& k) const override
    {
        const_cast<BSTEytPrefFour*>(this)->Base::freeze();
        const auto& a = Base::arr_;

        std::size_t i = 0;
        while (i < a.size()) {

            std::size_t q[32];           
            int front = 0, back = 0;
            q[back++] = i;

            for (int depth = 0; depth < 4; ++depth) {
                int levelCount = back - front;
                for (int n = 0; n < levelCount; ++n) {
                    std::size_t parent = q[front++];
                    std::size_t l = 2*parent + 1;
                    std::size_t r = l + 1;
                    pf(a, l); pf(a, r);
                    q[back++] = l;
                    q[back++] = r;
                }
            }

            if      (k == a[i]) return true;
            else if (k <  a[i]) i = 2*i + 1;
            else                i = 2*i + 2;
        }
        return false;
    }
};
\end{lstlisting}
\captionof{lstlisting}[\texttt{BSTEytPrefetchFour.h}]{Four-level software
prefetch variant (very aggressive look-ahead).}
\label{lst:bsteyt-pref-four-h}


\subsection{\texttt{include/BSTEytPrefProb.h}}
\label{secsec:bsteyt-pref-prob-h}
\begin{lstlisting}
#pragma once
#include "BSTEyt.h"
#include <cstddef>
#include <array>
#include <algorithm>
#include <cmath>

template<class Key, std::size_t Budget = 8>
class BSTEytPrefProb : public BSTEyt<Key> {
    using Base = BSTEyt<Key>;
    static_assert(Budget >= 2, "Budget must be at least two");


    static inline void pf(const std::vector<Key>& a, std::size_t idx)
    {
        if (idx < a.size()) __builtin_prefetch(&a[idx], 0, 1);
    }

    static void prefetchSubtree(const std::vector<Key>& a,
                                std::size_t            start,
                                std::size_t&           quota)
    {
        if (quota == 0 || start >= a.size()) return;

        std::array<std::size_t, Budget> q{};
        std::size_t front = 0, back = 0;
        q[back++] = start;

        while (front < back && quota) {
            std::size_t parent = q[front++];
            std::size_t l = 2 * parent + 1;
            std::size_t r = l + 1;

            if (l < a.size() && quota) {
                pf(a, l); --quota;
                if (quota && back < q.size()) q[back++] = l;
            }
            if (r < a.size() && quota) {
                pf(a, r); --quota;
                if (quota && back < q.size()) q[back++] = r;
            }
        }
    }

    mutable bool minmax_ready_ = false;
    mutable Key  min_key_{};
    mutable Key  max_key_{};

    [[gnu::always_inline]] inline void ensureMinMax() const
    {
        if (__builtin_expect(minmax_ready_, 1)) [[likely]]
            return;

        const auto& a = Base::arr_;
        if (__builtin_expect(a.empty(), 0)) [[unlikely]] return;                      
        auto [mn, mx] = std::minmax_element(a.begin(), a.end());
        min_key_ = *mn;
        max_key_ = *mx;
        minmax_ready_ = true;
    }

public:
    bool contains(const Key& k) const override
    {
        const_cast<BSTEytPrefProb*>(this)->Base::freeze();
        const auto& a = Base::arr_;
        if (a.empty()) return false;

        ensureMinMax();

        double ratio = (max_key_ == min_key_)
                     ? 0.5
                     : double(k - min_key_) / double(max_key_ - min_key_);
        ratio = std::clamp(ratio, 0.0, 1.0);

        std::size_t spent = 0;
        std::size_t i = 0;                
        std::size_t l = 1;                  
        std::size_t r = 2;                  

        if (l < a.size()) { pf(a, l); ++spent; }
        if (r < a.size()) { pf(a, r); ++spent; }

        std::size_t remaining = (Budget > spent) ? Budget - spent : 0;
        std::size_t left_budget  = std::size_t(std::round(remaining * (1.0 - ratio)));
        if (left_budget > remaining) left_budget = remaining;    
        std::size_t right_budget = remaining - left_budget;

        prefetchSubtree(a, l, left_budget);
        prefetchSubtree(a, r, right_budget);

        while (i < a.size()) {
            const Key& key = a[i];
            if (k == key) return true;
            i = (k < key) ? 2 * i + 1 : 2 * i + 2;
        }
        return false;
    }
};
\end{lstlisting}
\captionof{lstlisting}[\texttt{BSTEytPrefProb.h}]{Probabilistic,
path-biased prefetcher (\texttt{Budget}-controlled).}
\label{lst:bsteyt-pref-prob-h}


\subsection{\texttt{include/PerfCounters.h}}
\label{secsec:perfcounters-h}
\begin{lstlisting}
#pragma once
#include <linux/perf_event.h>
#include <cstdint>
#include <sys/syscall.h>
#include <sys/ioctl.h>
#include <unistd.h>
#include <cstring>
#include <cerrno>
#include <stdexcept>

class PerfCounters {
    int fd_refs_{-1},     fd_miss_{-1}; 
    int fd_l1_refs_{-1},  fd_l1_miss_{-1};
    int fd_l2_refs_{-1},  fd_l2_miss_{-1};
    int fd_l3_refs_{-1},  fd_l3_miss_{-1};
    int fd_br_{-1},       fd_br_miss_{-1};  

    long long refs_{0},  miss_{0};
    long long l1_refs_{0}, l1_miss_{0};
    long long l2_refs_{0}, l2_miss_{0};
    long long l3_refs_{0}, l3_miss_{0};
    long long br_{0},     br_miss_{0};

    static int open(uint32_t type, uint64_t cfg)
    {
        perf_event_attr pea{};
        pea.type           = type;
        pea.size           = sizeof(pea);
        pea.config         = cfg;
        pea.disabled       = 1;
        pea.exclude_kernel = 0;
        pea.exclude_hv     = 1;

        int fd = syscall(__NR_perf_event_open, &pea, 0, -1, -1, 0);
        if (fd == -1)
            throw std::runtime_error{"perf_event_open: " + std::string(strerror(errno))};
        return fd;
    }
    static int openCache(uint64_t id, uint64_t result)
    {
        uint64_t cfg = id
                     | (PERF_COUNT_HW_CACHE_OP_READ << 8)
                     | (result                       << 16);
        return open(PERF_TYPE_HW_CACHE, cfg);
    }
    static int tryOpenCache(uint64_t id, uint64_t result)
    {
        try { return openCache(id, result); }
        catch (...) { return -1; }
    }
public:
    PerfCounters()
    {
        fd_refs_  = open(PERF_TYPE_HARDWARE, PERF_COUNT_HW_CACHE_REFERENCES);
        fd_miss_  = open(PERF_TYPE_HARDWARE, PERF_COUNT_HW_CACHE_MISSES);

        fd_l1_refs_ = openCache(PERF_COUNT_HW_CACHE_L1D,
                                PERF_COUNT_HW_CACHE_RESULT_ACCESS);
        fd_l1_miss_ = openCache(PERF_COUNT_HW_CACHE_L1D,
                                PERF_COUNT_HW_CACHE_RESULT_MISS);

    #if defined(PERF_COUNT_HW_CACHE_L2)
        fd_l2_refs_ = tryOpenCache(PERF_COUNT_HW_CACHE_L2,
                                   PERF_COUNT_HW_CACHE_RESULT_ACCESS);
        fd_l2_miss_ = tryOpenCache(PERF_COUNT_HW_CACHE_L2,
                                   PERF_COUNT_HW_CACHE_RESULT_MISS);
    #elif defined(PERF_COUNT_HW_CACHE_L2D)
        fd_l2_refs_ = tryOpenCache(PERF_COUNT_HW_CACHE_L2D,
                                   PERF_COUNT_HW_CACHE_RESULT_ACCESS);
        fd_l2_miss_ = tryOpenCache(PERF_COUNT_HW_CACHE_L2D,
                                   PERF_COUNT_HW_CACHE_RESULT_MISS);
    #endif

        fd_l3_refs_ = openCache(PERF_COUNT_HW_CACHE_LL,
                                PERF_COUNT_HW_CACHE_RESULT_ACCESS);
        fd_l3_miss_ = openCache(PERF_COUNT_HW_CACHE_LL,
                                PERF_COUNT_HW_CACHE_RESULT_MISS);

        fd_br_      = open(PERF_TYPE_HARDWARE,
                           PERF_COUNT_HW_BRANCH_INSTRUCTIONS);
        fd_br_miss_ = open(PERF_TYPE_HARDWARE,
                           PERF_COUNT_HW_BRANCH_MISSES);
    }
    ~PerfCounters()
    {
        for (int fd : {fd_refs_, fd_miss_,
                       fd_l1_refs_, fd_l1_miss_,
                       fd_l2_refs_, fd_l2_miss_,
                       fd_l3_refs_, fd_l3_miss_,
                       fd_br_, fd_br_miss_})
            if (fd != -1) close(fd);
    }

    void start() const
    {
        for (int fd : {fd_refs_, fd_miss_,
                       fd_l1_refs_, fd_l1_miss_,
                       fd_l2_refs_, fd_l2_miss_,
                       fd_l3_refs_, fd_l3_miss_,
                       fd_br_, fd_br_miss_})
            if (fd != -1) {
                ioctl(fd, PERF_EVENT_IOC_RESET, 0);
                ioctl(fd, PERF_EVENT_IOC_ENABLE, 0);
            }
    }
    void stop()
    {
        for (int fd : {fd_refs_, fd_miss_,
                       fd_l1_refs_, fd_l1_miss_,
                       fd_l2_refs_, fd_l2_miss_,
                       fd_l3_refs_, fd_l3_miss_,
                       fd_br_, fd_br_miss_})
            if (fd != -1) ioctl(fd, PERF_EVENT_IOC_DISABLE, 0);

        auto rd = [](int fd, long long& dst){
            if (fd != -1 && read(fd, &dst, sizeof dst) != sizeof dst)
                throw std::runtime_error("read perf counter failed");
        };
        rd(fd_refs_, refs_);      rd(fd_miss_, miss_);
        rd(fd_l1_refs_, l1_refs_); rd(fd_l1_miss_, l1_miss_);
        rd(fd_l2_refs_, l2_refs_); rd(fd_l2_miss_, l2_miss_);
        rd(fd_l3_refs_, l3_refs_); rd(fd_l3_miss_, l3_miss_);
        rd(fd_br_, br_);           rd(fd_br_miss_, br_miss_);
    }

    long long refs()  const { return refs_;  }
    long long misses()const { return miss_;  }

    long long l1_refs()   const { return l1_refs_;   }
    long long l1_misses() const { return l1_miss_;   }

    long long l2_refs()   const { return l2_refs_;   }
    long long l2_misses() const { return l2_miss_;   }

    long long l3_refs()   const { return l3_refs_;   }
    long long l3_misses() const { return l3_miss_;   }

    long long branches()       const { return br_;       }
    long long branch_misses()  const { return br_miss_;  }
};
\end{lstlisting}
\captionof{lstlisting}[\texttt{PerfCounters.h}]{Wrapper around Linux
interface for hardware counter sampling.}
\label{lst:perfcounters-h}


\subsection{\texttt{Makefile}}
\label{secsec:makefile}
\begin{lstlisting}
CXX      := g++
CXXFLAGS := -std=c++20 -O3 -march=native -DNDEBUG -Iinclude -Wall -Wextra

SRC  := $(wildcard src/*.cpp)
OBJ  := $(SRC:src/%.cpp=build/%.o)
BIN  := bst-bench

TEST_SRC   := $(wildcard test/*.cpp)
TEST_OBJ   := $(TEST_SRC:test/%.cpp=build/test/%.o)
TEST_BIN   := bst-tests

TEST_CXXFLAGS := $(filter-out -DNDEBUG,$(CXXFLAGS))

.PHONY: all
all: $(BIN)

.PHONY: test
test: $(TEST_BIN)
	./$(TEST_BIN)

$(TEST_BIN): $(TEST_OBJ)
	$(CXX) $(TEST_CXXFLAGS) $^ -o $@

$(BIN): $(OBJ)
	$(CXX) $(CXXFLAGS) $^ -o $@

build/test/%.o: test/%.cpp
	@mkdir -p $(@D)
	$(CXX) $(TEST_CXXFLAGS) -c $< -o $@

build/%.o: src/%.cpp
	@mkdir -p $(@D)
	$(CXX) $(CXXFLAGS) -c $< -o $@

.PHONY: clean
clean:
	rm -rf build $(BIN) $(TEST_BIN)

\end{lstlisting}
\captionof{lstlisting}[\texttt{Makefile}]{Build rules: default target builds
\texttt{bst-bench}; \texttt{make test} builds+executes unit tests.}
\label{lst:makefile}



\subsection{\texttt{run\_bench.sh}}
\label{secsec:run-bench-sh}
\begin{lstlisting}
set -euo pipefail

usage() {
  echo "Usage: $0 <base_json> <n_min> <n_max> <step> [--link-q] [--cpu ID] [--nice N]"
  exit 1
}

if [[ $# -lt 4 ]]; then usage; fi

BASE_CFG=$1; shift
N_MIN=$1;   shift
N_MAX=$1;   shift
STEP=$1;    shift

CPU_ID=0          
NICE_VAL=-20      
LINK_Q=false

while [[ $# -gt 0 ]]; do
  case $1 in
    --link-q) LINK_Q=true; shift ;;
    --cpu)    CPU_ID=$2;  shift 2 ;;
    --nice)   NICE_VAL=$2; shift 2 ;;
    *)        usage ;;
  esac
done

echo "# Running on CPU $CPU_ID  (nice $NICE_VAL)"
echo "# n from $N_MIN to $N_MAX in steps of $STEP"
$LINK_Q && echo "# q linked to n"

make

IMPLS=(
  "BST_VEB" "BST_EYT" "BST_EYT_PREF"
  "BST_EYT_PREF_TWO" "BST_EYT_PREF_THREE"
  "BST_EYT_PREF_FOUR" "BST_EYT_PREF_PROB"
)

TMP=$(mktemp)
cleanup() { rm -f "$TMP"; }
trap cleanup EXIT

for (( N=N_MIN; N<=N_MAX; N+=STEP )); do
  if $LINK_Q; then
    jq --argjson n "$N" '.n=$n | .q=$n' "$BASE_CFG" > "$TMP"
  else
    jq --argjson n "$N" '.n=$n' "$BASE_CFG"        > "$TMP"
  fi

  for impl in "${IMPLS[@]}"; do
  nice -n "$NICE_VAL" chrt --fifo 99 taskset -c "$CPU_ID" ./bst-bench "$TMP" "$impl"
done
done


# sudo bash scale_bench.sh data/base.json 500000 10000000 500000  --cpu 2 
\end{lstlisting}
\captionof{lstlisting}[\texttt{run\_bench.sh}]{Helper script: compiles and
launches \texttt{bst-bench} for all variants. Sets up the system to give high priority to the benchmark process.}
\label{lst:run-bench-sh}


\subsection{\texttt{src/benchmark.cpp}}
\label{secsec:src-benchmark-cpp}
\begin{lstlisting}
#include "IBST.h"
#include "BSTVEB.h"
#include "BSTEyt.h"
#include "BSTEytPrefetch.h"
#include "BSTEytPrefetchTwo.h"
#include "BSTEytPrefetchThree.h"
#include "BSTEytPrefetchFour.h"
#include "BSTEytPrefetchProb.h"
#include "PerfCounters.h"
#include <vector>
#include <random>
#include <iostream>
#include <iomanip>
#include <fstream>
#include <functional>
#include <chrono>
#include "util/json.hpp"

using json  = nlohmann::json;
using Clock = std::chrono::steady_clock;

struct Metrics {
    long long ns = 0, ops = 0;

    long long c_refs = 0, c_miss = 0;
    long long l1_refs = 0, l1_miss = 0;
    long long l2_refs = 0, l2_miss = 0;
    long long l3_refs = 0, l3_miss = 0;

    long long branches = 0, br_miss = 0;
};

template<class Key>
Metrics benchOnce(IBST<Key>& tree,
                  const std::vector<Key>& lookups,
                  const std::vector<Key>& inserts,
                  bool       measure_construction)
{
    for (const auto& k : inserts) tree.insert(k);

    if (!measure_construction && !lookups.empty()) (void)tree.contains(lookups[0]);

    PerfCounters pc; pc.start();
    auto t0 = Clock::now();
    for (const auto& k : lookups) (void)tree.contains(k);
    auto t1 = Clock::now();
    pc.stop();

    Metrics m;
    m.ns       = std::chrono::duration_cast<std::chrono::nanoseconds>(t1 - t0).count();
    m.ops      = lookups.size();

    m.c_refs   = pc.refs();       m.c_miss   = pc.misses();
    m.l1_refs  = pc.l1_refs();    m.l1_miss  = pc.l1_misses();
    m.l2_refs  = pc.l2_refs();    m.l2_miss  = pc.l2_misses();
    m.l3_refs  = pc.l3_refs();    m.l3_miss  = pc.l3_misses();
    m.branches = pc.branches();   m.br_miss  = pc.branch_misses();
    return m;
}

using Factory = std::function<std::unique_ptr<IBST<int>>()>;
struct Variant { std::string name; Factory make; };

const std::vector<Variant> variants = {
    {"BST_VEB",      [] { return std::make_unique<BSTVEB<int>>(); }},
    {"BST_EYT",      [] { return std::make_unique<BSTEyt<int>>(); }},
    {"BST_EYT_PREF", [] { return std::make_unique<BSTEytPref<int>>(); }},
    {"BST_EYT_PREF_TWO", [] { return std::make_unique<BSTEytPrefTwo<int>>(); }},
    {"BST_EYT_PREF_THREE", [] { return std::make_unique<BSTEytPrefThree<int>>(); }},
    {"BST_EYT_PREF_FOUR", [] { return std::make_unique<BSTEytPrefFour<int>>(); }},
    {"BST_EYT_PREF_PROB", [] { return std::make_unique<BSTEytPrefProb<int>>(); }},
};

void runExperiment(int n, int q, int T, bool csv,
                   const Factory& make, unsigned seed,
                   const std::string& impl, bool measure_construction)
{
    std::mt19937 rng(seed);
    std::uniform_int_distribution<int> dist(1, n * 10);

    std::vector<int> inserts(n);
    for (int& x : inserts) x = dist(rng);
    std::vector<int> lookups(q);
    for (int& x : lookups) x = dist(rng);

    long long acc_ns = 0,
              acc_c_refs = 0,  acc_c_miss = 0,
              acc_l1_refs = 0, acc_l1_miss = 0,
              acc_l2_refs = 0, acc_l2_miss = 0,
              acc_l3_refs = 0, acc_l3_miss = 0,
              acc_br = 0,      acc_br_miss = 0;

    std::size_t bytes_used = 0;

    for (int t = 0; t < T; ++t) {
        auto tree = make();
        Metrics m = benchOnce(*tree, lookups, inserts, measure_construction);

        acc_ns      += m.ns;
        acc_c_refs  += m.c_refs;  acc_c_miss  += m.c_miss;
        acc_l1_refs += m.l1_refs; acc_l1_miss += m.l1_miss;
        acc_l2_refs += m.l2_refs; acc_l2_miss += m.l2_miss;
        acc_l3_refs += m.l3_refs; acc_l3_miss += m.l3_miss;
        acc_br      += m.branches;acc_br_miss += m.br_miss;

        if (t == 0) bytes_used = tree->size_bytes();
    }

    auto avgLL = [T](long long v){ return double(v) / T; };

    double avg_ns       = avgLL(acc_ns),   ns_per_op = avg_ns / q, avg_s = avg_ns / 1e9;
    double avg_c_refs   = avgLL(acc_c_refs),  avg_c_miss = avgLL(acc_c_miss);
    double avg_l1_refs  = avgLL(acc_l1_refs), avg_l1_miss = avgLL(acc_l1_miss);
    double avg_l2_refs  = avgLL(acc_l2_refs), avg_l2_miss = avgLL(acc_l2_miss);
    double avg_l3_refs  = avgLL(acc_l3_refs), avg_l3_miss = avgLL(acc_l3_miss);
    double avg_br       = avgLL(acc_br),      avg_br_miss = avgLL(acc_br_miss);

    auto rate = [](double miss, double ref){ return ref ? miss / ref : 0.0; };

    double miss_per_op  = avg_c_miss / q;
    double miss_rate    = rate(avg_c_miss, avg_c_refs);
    double l1_rate      = rate(avg_l1_miss, avg_l1_refs);
    double l2_rate      = rate(avg_l2_miss, avg_l2_refs);
    double l3_rate      = rate(avg_l3_miss, avg_l3_refs);
    double br_rate      = rate(avg_br_miss, avg_br);

    double bytes_mb     = bytes_used / 1024.0 / 1024.0;

    if (csv) {
        std::cout << impl << ','
                  << n << ',' << q << ','
                  << avg_ns << ',' << avg_s << ','
                  << ns_per_op << ','
                  << avg_c_refs << ',' << avg_c_miss << ','
                  << miss_per_op << ',' << miss_rate << ','
                  << bytes_used << ','
                  << avg_l1_refs << ',' << avg_l1_miss << ',' << l1_rate << ','
                  << avg_l2_refs << ',' << avg_l2_miss << ',' << l2_rate << ','
                  << avg_l3_refs << ',' << avg_l3_miss << ',' << l3_rate << ','
                  << avg_br << ',' << avg_br_miss << ',' << br_rate
                  << '\n';
    } else {
    std::cout << std::left << std::fixed <<  std::setprecision(2)
              << std::setw(22)  << impl
              << std::setw(8)  << n
              << std::setw(8)  << q
              << std::setw(14) << std::setprecision(0) << avg_ns
              << std::setw(8)  << std::setprecision(2) << avg_s
              << std::setw(14) << std::setprecision(6) << ns_per_op
              << std::setw(12) << std::setprecision(0) << avg_c_refs
              << std::setw(12) << avg_c_miss
              << std::setw(12) << std::setprecision(2) << miss_per_op
              << std::setw(10) << std::setprecision(6) << miss_rate
              << std::setw(6)  << std::setprecision(1) << bytes_mb
              << std::setw(12) << std::setprecision(0) << avg_l1_refs
              << std::setw(12) << avg_l1_miss
              << std::setw(10) << std::setprecision(6) << l1_rate
              << std::setw(12) << std::setprecision(0) << avg_l2_refs
              << std::setw(12) << avg_l2_miss
              << std::setw(10) << std::setprecision(6) << l2_rate
              << std::setw(12) << std::setprecision(0) << avg_l3_refs
              << std::setw(12) << avg_l3_miss
              << std::setw(10) << std::setprecision(6) << l3_rate
              << std::setw(12) << std::setprecision(0) << avg_br
              << std::setw(12) << avg_br_miss
              << std::setw(10) << std::setprecision(6) << br_rate
              << '\n';
}
}

int main(int argc, char* argv[])
{
    int  n = 10000, q = 10000, T = 1;
    bool csv = false;
    unsigned seed = 42;
    std::string impl = "ALL";
    bool measure_construction = true;

    if (argc >= 2) {
        std::ifstream in(argv[1]);
        if (!in) { std::cerr << "Cannot open " << argv[1] << '\n'; return 1; }
        json cfg; in >> cfg;
        if (cfg.contains("n"))    n    = cfg["n"];
        if (cfg.contains("q"))    q    = cfg["q"];
        if (cfg.contains("T"))    T    = cfg["T"];
        if (cfg.contains("csv"))  csv  = cfg["csv"];
        if (cfg.contains("seed")) seed = cfg["seed"];
        if (cfg.contains("impl")) impl = cfg["impl"];
        if (cfg.contains("measure_construction")) measure_construction = cfg["measure_construction"];
    }
    if (argc == 3) impl = argv[2];

    if (!csv) {
    std::cout << std::left
              << std::setw(22)  << "impl"
              << std::setw(8)  << "n"
              << std::setw(8)  << "q"
              << std::setw(14) << "total_ns"
              << std::setw(8)  << "total_s"
              << std::setw(14) << "ns/search"
              << std::setw(12) << "c_refs"
              << std::setw(12) << "c_miss"
              << std::setw(12) << "miss/sea"
              << std::setw(10) << "miss_rate"
              << std::setw(6)  << "MB"
              << std::setw(12) << "L1_refs"
              << std::setw(12) << "L1_miss"
              << std::setw(10) << "L1_rate"
              << std::setw(12) << "L2_refs"
              << std::setw(12) << "L2_miss"
              << std::setw(10) << "L2_rate"
              << std::setw(12) << "L3_refs"
              << std::setw(12) << "L3_miss"
              << std::setw(10) << "L3_rate"
              << std::setw(12) << "branches"
              << std::setw(12) << "br_miss"
              << std::setw(10) << "br_rate"
              << '\n'
              << std::string(255, '-') << '\n';
}else {
        std::cout << "impl,n,q,total_ns,total_s,ns_per_search,"
                     "cache_refs,cache_misses,misses_per_search,"
                     "miss_rate,bytes,"
                     "l1_refs,l1_misses,l1_rate,"
                     "l2_refs,l2_misses,l2_rate,"
                     "l3_refs,l3_misses,l3_rate,"
                     "branches,branch_misses,branch_rate\n";
    }

    for (const auto& v : variants) {
        if (impl != "ALL" && impl != v.name) continue;
        runExperiment(n, q, T, csv, v.make, seed, v.name, measure_construction);
    }
    return 0;
}

\end{lstlisting}
\captionof{lstlisting}[\texttt{benchmark.cpp}]{Main benchmarking driver:
parses JSON config, samples perf counters, and prints results.}
\label{lst:src-benchmark-cpp}


\subsection{\texttt{test/ContainsTest.cpp}}
\label{secsec:test-containstest-cpp}
\begin{lstlisting}
#include <cassert>
#include <vector>
#include <iostream>

#include "../include/BSTVEB.h"
#include "../include/BSTEyt.h"
#include "../include/BSTEytPrefetch.h"
#include "../include/BSTEytPrefetchTwo.h"
#include "../include/BSTEytPrefetchThree.h"
#include "../include/BSTEytPrefetchFour.h"
#include "../include/BSTEytPrefetchProb.h"

template<class Tree>
void sanity_check(std::size_t N = 1'024)
{
    Tree t;
    for (std::size_t i = 0; i < N; ++i) t.insert(static_cast<int>(i * 2));

    for (std::size_t i = 0; i < N; ++i) {
        int present = static_cast<int>(i * 2);
        int absent  = present + 1;
        assert(t.contains(present) && "key should be present");
        assert(!t.contains(absent) && "key should be absent");
    }
}

int main()
{
    sanity_check< BSTVEB<int>           >();
    sanity_check< BSTEyt<int>           >();
    sanity_check< BSTEytPref<int>       >();
    sanity_check< BSTEytPrefTwo<int>    >();
    sanity_check< BSTEytPrefThree<int>  >();
    sanity_check< BSTEytPrefFour<int>   >();
    sanity_check< BSTEytPrefProb<int>   >();

    std::cout << "all imps contains() tests passed\n";
    return 0;
}

\end{lstlisting}
\captionof{lstlisting}[\texttt{ContainsTest.cpp}]{Unit tests verifying that
each BST variant correctly answers \texttt{contains()} queries.}
\label{lst:test-containstest-cpp}

\subsection{\texttt{data/large.json}}
\label{secsec:data-large-json}
\begin{lstlisting}
{
  "n"   : 10000000,
  "q"   : 10000000,
  "T"   : 5,
  "csv" : false,
  "seed": 123,
  "measure_construction": false
}
\end{lstlisting}
\captionof{lstlisting}[\texttt{large.json}]{Benchmark configuration for the
largest test instance (\texttt{n=10\,000\,000}).}
\label{lst:data-large-json}

\subsection{\texttt{data/medium.json}}
\label{secsec:data-medium-json}
\begin{lstlisting}
{
  "n"   : 5000000,
  "q"   : 5000000,
  "T"   : 5,
  "csv" : false,
  "seed": 123,
  "measure_construction": false
}
\end{lstlisting}
\captionof{lstlisting}[\texttt{medium.json}]{Benchmark configuration for the
medium-sized instance (\texttt{n=5\,000\,000}).}
\label{lst:data-medium-json}

\subsection{\texttt{data/small.json}}
\label{secsec:data-small-json}
\begin{lstlisting}
{
  "n"   : 100000,
  "q"   : 100000,
  "T"   : 5,
  "csv" : false,
  "seed": 123,
  "measure_construction": false
}
\end{lstlisting}
\captionof{lstlisting}[\texttt{small.json}]{Benchmark configuration for the
small instance (\texttt{n=100\,000}).}
\label{lst:data-small-json}

\subsection{\texttt{data/test.json}}
\label{secsec:data-test-json}
\begin{lstlisting}
{
  "n"   : 1000000,
  "q"   : 1000000,
  "T"   : 1,
  "csv" : false,
  "seed": 123,
  "measure_construction": false
}
\end{lstlisting}
\captionof{lstlisting}[\texttt{test.json}]{Quick-run configuration used in
unit-test and CI pipelines.}
\label{lst:data-test-json}

\subsection{\texttt{scale\_bench.sh}}
\label{secsec:scale}
\begin{lstlisting}
set -euo pipefail

usage() {
  echo "Usage: $0 <base_json> <n_min> <n_max> <step> [--link-q] [--cpu ID] [--nice N]"
  exit 1
}

if [[ $# -lt 4 ]]; then usage; fi

BASE_CFG=$1; shift
N_MIN=$1;   shift
N_MAX=$1;   shift
STEP=$1;    shift

CPU_ID=0          
NICE_VAL=-20      
LINK_Q=false

while [[ $# -gt 0 ]]; do
  case $1 in
    --link-q) LINK_Q=true; shift ;;
    --cpu)    CPU_ID=$2;  shift 2 ;;
    --nice)   NICE_VAL=$2; shift 2 ;;
    *)        usage ;;
  esac
done

echo "# Running on CPU $CPU_ID  (nice $NICE_VAL)"
echo "# n from $N_MIN to $N_MAX in steps of $STEP"
$LINK_Q && echo "# q linked to n"

make

IMPLS=(
  "BST_EYT_PREF_THREE_NOIC"
)

TMP=$(mktemp)
cleanup() { rm -f "$TMP"; }
trap cleanup EXIT

for (( N=N_MIN; N<=N_MAX; N+=STEP )); do
  if $LINK_Q; then
    jq --argjson n "$N" '.n=$n | .q=$n' "$BASE_CFG" > "$TMP"
  else
    jq --argjson n "$N" '.n=$n' "$BASE_CFG"        > "$TMP"
  fi

  for impl in "${IMPLS[@]}"; do
  nice -n "$NICE_VAL" chrt --fifo 99 taskset -c "$CPU_ID" ./bst-bench "$TMP" "$impl"
done
done


# sudo bash scale_bench.sh data/base.json 500000 10000000 500000  --cpu 2 
\end{lstlisting}
\captionof{lstlisting}[\texttt{scale\_bench.sh}]{\texttt{scale\_bench.sh} for running the large benchmark reported.}
\label{lst:scale}