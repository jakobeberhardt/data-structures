\begin{figure}[ht]
\centering
\begin{tabular}{ c c c }
\adjustbox{valign=m}{%
\begin{minipage}{0.25\textwidth}
\begin{lstlisting}[language=C++, basicstyle=\ttfamily\small]
if (x > 0) y = 10;
else y = 20;
\end{lstlisting}
\end{minipage}
}
&
\adjustbox{valign=m}{%
\begin{tikzpicture}
    \draw[->, line width=0.5pt] (0,0.5) -- (2, 1.5);
    \draw[->, line width=0.5pt] (0,-0.5) -- (2, -1.5);
\end{tikzpicture}
}
&
\adjustbox{valign=m}{%
\begin{minipage}{0.4\textwidth}
\textbf{Before If-Conversion:}
\begin{lstlisting}[basicstyle=\ttfamily\small]
    cmp eax, 0
    jg greater
    mov ebx, 20
    jmp end
greater:
    mov ebx, 10
end:
\end{lstlisting}

\vspace{0.8cm}

\textbf{With If-Conversion:}
\begin{lstlisting}[basicstyle=\ttfamily\small]
    mov ebx, 20
    cmp eax, 0
    mov ecx, 10
    cmovg ebx, ecx
\end{lstlisting}
\end{minipage}
}
\end{tabular}
\caption[If-conversion example with x86 \texttt{cmov}]{If-conversion applied to a conditional assignment in C++, showing both the original x86 assembly with branching and the optimized version using the conditional move (\texttt{cmov}) instruction.}
\label{fig:ifconv}
\end{figure}
